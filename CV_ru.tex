%%%%%%%%%%%%%%%%%%%%%%%%%%%%%%%%%%%%%%%%%
% "ModernCV" CV and Cover Letter
% LaTeX Template
% Version 1.11 (19/6/14)
%
% This template has been downloaded from:
% http://www.LaTeXTemplates.com
%
% Original author:
% Xavier Danaux (xdanaux@gmail.com)
%
% License:
% CC BY-NC-SA 3.0 (http://creativecommons.org/licenses/by-nc-sa/3.0/)
%
% Important note:
%----------------------------------------------------------------------------------------
%	PACKAGES AND OTHER DOCUMENT CONFIGURATIONS
%----------------------------------------------------------------------------------------

%\documentclass[12pt,a4paper,sans]{moderncv} % Font sizes: 10, 11, or 12; paper sizes: a4paper, letterpaper, a5paper, legalpaper, executivepaper or landscape; font families: sans or roman

\documentclass[12pt,a4paper]{moderncv}
\usepackage[russian]{babel}
%\usepackage[T2A]{fontenc}

\moderncvstyle{classic}
\moderncvcolor{black} % CV color - options include: 'blue' or 'black'

\usepackage{lipsum} % Used for inserting dummy 'Lorem ipsum' text into the template

\usepackage[scale=0.75]{geometry} % Reduce document margins
%\setlength{\hintscolumnwidth}{3cm} % Uncomment to change the width of the dates column
%\setlength{\makecvtitlenamewidth}{10cm} % For the 'classic' style, uncomment to adjust the width of the space allocated to your name

\usepackage[utf8]{inputenc}
\usepackage{wasysym}

%\renewcommand{\rmdefault}{cmr} % Шрифт с засечками  
%\renewcommand{\sfdefault}{cmss} % Шрифт без засечек  
%\renewcommand{\ttdefault}{cmtt} % Моноширинный шрифт %%%%

%----------------------------------------------------------------------------------------
%	NAME AND CONTACT INFORMATION SECTION
%----------------------------------------------------------------------------------------

\firstname{Михаил} % Your first name
\familyname{\newlineКормышов} % Your last name

% All information in this block is optional, comment out any lines you don't need
%\title{Curriculum Vitae}
\address{Гражданский проспект, 117а-456}{Санкт-Петербург}
\mobile{(999) 442 3087}
%mobile{(902) 942 7863}
%\phone{(000) 111 1112}
%\fax{(000) 111 1113}
\email{kormyshov@gmail.com}
\homepage{k-mihey.livejournal.com}{k-mihey.livejournal.com} % The first argument is the url for the clickable link, the second argument is the url displayed in the template - this allows special characters to be displayed such as the tilde in this example
%\extrainfo{additional information}
%\photo[70pt][0.4pt]{picture} % The first bracket is the picture height, the second is the thickness of the frame around the picture (0pt for no frame)
%\quote{"A witty and playful quotation" - John Smith}

%----------------------------------------------------------------------------------------

\begin{document}

\makecvtitle % Print the CV title

%----------------------------------------------------------------------------------------
%	EDUCATION SECTION
%----------------------------------------------------------------------------------------

\section{Образование}

\cventry{2003--2008}{Высшее}{Сибирский федеральный университет}{Красноярск}{\newlineспециальность: математик, системный программист}{}  % Arguments not required can be left empty

\section{Дипломная работа}

\cvitem{Название}{\emph{Модуль для системы дистанционного образования «Moodle» автопроверки корректности работы программы на серии тестов}}
\cvitem{Науч. рук.}{Старший преподаватель А.В. Бархатов}
\cvitem{Описание}{В рамках данной работы была разработана система автоматической проверки решений задач по программированию. Также был написан модуль к распространенной и применяемой в нашем университете СДО «Moodle» для взаимодействия преподавателей и студентов с системой проверки программ.}

%----------------------------------------------------------------------------------------
%	WORK EXPERIENCE SECTION
%----------------------------------------------------------------------------------------

\section{Опыт}

\subsection{Работа}

\cventry{2013--Настоящее время}{Генеральный директор}{\textsc{ООО <<Тренинг-центр <<Профит>>>>}}{Красноярск}{}{Основатель и ген.директор, помимо управленческой деятельности проводил занятия по спортивному программированию и подготовке к ЕГЭ по информатике.
\newline{}\newline{}
Результаты:
\begin{itemize}
  \item 2013-14 учебный год
  \begin{itemize}
    \item 19 человек прошли на региональный этап ВсОШ (всего мест 60), трое стали победителями (в каждой параллели) и четверо -- призёрами
    \item чемпионы Восточной Сибири в рамках полуфинала ВКОШП, а также 2,3,5 и 10 места у команд, которые прошли в финал ВКОШП
    \item призёр на заключительном этапе ВсОШ по информатике и ИКТ
    \item дипломы II и III степени на Индивидуальной олимпиаде по информатике и программированию
  \end{itemize}
  \item 2014-15 учебный год
  \begin{itemize}
    \item 23 человека прошли на региональный этап ВсОШ (всего мест 80), четверо стали победителями (в 11ых классах двое) и четверо -- призёрами
    \item чемпионы Восточной Сибири в рамках полуфинала ВКОШП, а также 4-6, 9, 13 места у команд, которые прошли в финал ВКОШП
    \item 2 призёра на заключительном этапе ВсОШ по информатике и ИКТ
    \item диплом III степени на Индивидуальной олимпиаде по информатике и программированию
  \end{itemize}
\end{itemize}
\newline{}\newline{}\newline{}
В июне 2014 года выпущена книга <<Справочник спортивного программиста (часть 1)>>.
\newline{}\newline{}
В августе 2014 года было выпущено приложение для ВКонтакте <<Тренажер ЕГЭ по информатике>>.
\newline{}
}

%------------------------------------------------

\cventry{2013--Настоящее время}{Инженер лаборатории математики и информатики}{\textsc{ФГАОУ ВПО Сибирский федеральный университет}}{Красноярск}{}{Выполнял роль тренера по спортивному программированию команд Института математики и фундаментальной информатики. Команды добились лучших результатов именно в этот период:
\begin{itemize}
  \item 2013 год
  \begin{itemize}
    \item чемпионы Восточной Сибири (вторая команда на 4-ом месте)
    \item второе и третье место в личном зачёте и второе в командном на региональной олимпиаде по программированию (г. Новокузнецк)
    \item 30 место на NEERC ACM ICPC
  \end{itemize}
  \item 2014 год
  \begin{itemize}
    \item чемпионы Восточной Сибири (вторая команда на 6-ом месте)
    \item 39 место на NEERC ACM ICPC
  \end{itemize}
\end{itemize}
\newline{}
}

%------------------------------------------------

\cventry{2013--2015}{Тренер по спортивному программированию}{\textsc{МБОУ <<Лицей~№174>>}}{Зеленогорск}{}{Проводил интенсивные недельные курсы по алгоритмам и структурам данных раз в месяц. За это время учащиеся начали показывать результаты на олимпиадах регионального и Всероссийского уровня. В том числе: 2 победителя и 2 призёра регионального этапа Всероссийской олимпиады школьников по информатике и ИКТ; и призёр заключительного этапа (впервые в городе).
\newline{}}

%------------------------------------------------

\cventry{лето 2015, 3 смена}{Воспитатель}{\textsc{СОК <<Зелёные горки>>}}{Красноярск}{}{Работал воспитателем на 6 экипаже (14-17 лет), который занял первое место по итогам смены. А также преподавателем информатики. За смену организовал и провёл Антинаучный конгресс и три интеллектуальных турнира: Что? Где? Когда?, Доминошка и Своя игра.
\newline{}}

%------------------------------------------------

\cventry{2010--2015}{Преподаватель информатики/математики, инструктор}{\textsc{Выездная школа <<Перспектива>>}}{Красноярск}{}{Принял участие в 20 сменах школы (в том числе 6 трёхнедельных). На всех был преподавателем информатики и тренером по спортивному программированию, на многих из них также выполнял роль инструктора на экипаже.
\newline{}}

%------------------------------------------------

\cventry{2009--2015}{Автор задач}{\textsc{Региональная предметно-методическая комиссия Красноярского края по информатике}}{}{}{
\newline{}}

%------------------------------------------------

\cventry{2004--2015}{Член жюри}{\textsc{Региональный этап Всероссийской олимпиады школьников по информатике и ИКТ}}{Красноярск}{}{
\newline{}}

%------------------------------------------------

\cventry{2004--2015}{Оргкомитет}{\textsc{Региональный этап Всероссийской олимпиады школьников}}{Красноярск}{}{
\newline{}}

%------------------------------------------------

\cventry{лето 2014, 3 смена}{Воспитатель}{\textsc{СОК <<Зелёные горки>>}}{Красноярск}{}{Работал воспитателем на 4 экипаже (14-17 лет). А также преподавателем информатики.
\newline{}}

%------------------------------------------------

\cventry{лето 2013, 2 смена}{Воспитатель}{\textsc{СОК <<Зелёные горки>>}}{Красноярск}{}{Работал воспитателем на 4 экипаже (14-17 лет). А также преподавателем информатики. За смену организовал и провёл Антинаучный конгресс и два интеллектуальных турнира: Что? Где? Когда? и Доминошка.
\newline{}}

%------------------------------------------------

\cventry{2010--2012}{Учитель информатики}{\textsc{КГОАУ <<Школа космонавтики>>}}{Железногорск}{}{Помимо уроков в 9-11 классах проводил курс <<Решение олимпиадных задач по информатике>>. Именно в этот период ученики школы показали лучшие результаты на олимпиадах по информатике и программированию: 2 победителя и призёр на региональном этапе ВсОШ по информатике и ИКТ, участник заключительного этапа в 2012 году.
\newline{}}

%------------------------------------------------

\cventry{2008--2012}{Педагог дополнительного образования}{\textsc{МОУ <<Межшкольный учебный комбинат~№4>>}}{Красноярск}{}{Тренировал школьников 9-11 классов для участия в олимпиадах по информатике.
\newline{}}

%------------------------------------------------

\cventry{2008--2012}{Тренер по спортивному программированию}{\textsc{МОУ <<ОУ Лицей~№7>>}}{Красноярск}{}{
\newline{}}

%------------------------------------------------

\cventry{июнь 2010}{Преподаватель информатики}{\textsc{Интенсивная школа <<Талант>>}}{Железногорск}{}{
\newline{}}

%------------------------------------------------

\cventry{2008--2010}{Учитель математики и информатики}{\textsc{МОУ <<ОУ Гимназия~№15>>}}{Красноярск}{}{
\newline{}}

%------------------------------------------------

\cventry{2005--2007}{Учитель информатики}{\textsc{МОУ СОШ~№146}}{Красноярск}{}{
\newline{}}

%------------------------------------------------

\subsection{Разное}

\cventry{2013--2014}{Автор задач}{Турнир по программированию}{Абакан}{}{Подготовил задачи для двух контестов и провёл разбор после соревнования.}

%----------------------------------------------------------------------------------------
%	AWARDS SECTION
%----------------------------------------------------------------------------------------

\section{Награды}

\cvitem{2013}{Лауреат премии главы города Красноярска <<Молодым талантам>> за высокие достижения в научно-учебной деятельности}
\cvitem{2010}{Благодарственное письмо Министерства образования и науки Красноярского края за вклад в реализацию направления <<Развитие системы поддержки талантливых детей>> национальной образовательной инициативы <<Наша новая школа>>}
\cvitem{2010}{Благодарственное письмо Министерства образования и науки Красноярского края за подготовку призёров и победителей регионального этапа Всероссийской олимпиады школьников}
\cvitem{2008}{Лауреат премии администрации Октябрьского района г.~Красноярска молодым талантам}

%----------------------------------------------------------------------------------------
%	COMPUTER SKILLS SECTION
%----------------------------------------------------------------------------------------

\section{Компьютерные навыки}

\cvitem{Basic}{Python, Java, Haskell, GIMP}
\cvitem{Intermediate}{HTML, JS, php, Linux, \LaTeX, LibreOffice}
\cvitem{Advanced}{C++, vim}

%----------------------------------------------------------------------------------------
%	COMMUNICATION SKILLS SECTION
%----------------------------------------------------------------------------------------

%\section{Communication Skills}

%\cvitem{2010}{Oral Presentation at the California Business Conference}
%\cvitem{2009}{Poster at the Annual Business Conference in Oregon}

%----------------------------------------------------------------------------------------
%	LANGUAGES SECTION
%----------------------------------------------------------------------------------------

\section{Знание языков}

\cvitemwithcomment{English}{Pre-Intermediate}{}

%----------------------------------------------------------------------------------------
%	COMPETITION SKILLS SECTION
%----------------------------------------------------------------------------------------

\section{Спортивное программирование}

\cvitem{codeforces}{kormyshov, 1976, max. 2145}
\cvitem{topcoder}{K\_Mihey, 1477, max. 1694}
\cvitem{ACM ICPC}{Чемпион Восточной Сибири в рамках четвертьфинала ACM ICPC'07}
\cvitem{КРОК}{7 место на чемпионате КРОК, г.~Москва, 2012г}
\cvitem{Софт-парад}{1 место (полный балл) в блиц-турнире регионального смотра-конкурса <<Софт-парад>>, г.~Красноярск, 2013г}
\cvitem{Google Code Jam}{Прохождение в третий раунд в 2012, 2013 и 2015 годах}
\cvitem{TopCoder Open (Algorithm)}{Прохождение во второй раунд в 2013, 2014 и 2015 годах}

%----------------------------------------------------------------------------------------
%	INTERESTS SECTION
%----------------------------------------------------------------------------------------

\section{Интересы}

\renewcommand{\listitemsymbol}{--~} % Changes the symbol used for lists

\cvlistdoubleitem{Нумизматика}{Настольные игры}
\cvlistdoubleitem{Коньки}{Гитара}
\cvlistitem{Железные дороги}

%----------------------------------------------------------------------------------------
%	COVER LETTER
%----------------------------------------------------------------------------------------

% To remove the cover letter, comment out this entire block

%\clearpage

%\recipient{HR Department}{Corporation\\123 Pleasant Lane\\12345 City, State} % Letter recipient
%\date{\today} % Letter date
%\opening{Dear Sir or Madam,} % Opening greeting
%\closing{Sincerely yours,} % Closing phrase
%\enclosure[Attached]{curriculum vit\ae{}} % List of enclosed documents

%\makelettertitle % Print letter title

%\lipsum[1-3] % Dummy text

%\makeletterclosing % Print letter signature

%----------------------------------------------------------------------------------------

\end{document}
